\documentclass[10pt,a4paper]{article}

% Encodage et langue
\usepackage[T1]{fontenc}
\usepackage[utf8]{inputenc}
\usepackage[french]{babel}

% Couleurs
\usepackage[table,usenames,dvipsnames]{xcolor}

% Graphiques et légendes
\usepackage{graphicx}
\usepackage{caption}

% NiceMatrix
\usepackage{nicematrix}

% Maths
\usepackage{amsmath}
\usepackage{amsfonts}
\usepackage{amssymb}

% Mise en page
\usepackage{geometry}

% Entêtes et pied de page
\usepackage{fancyhdr}
\usepackage{lastpage}

% Forest
\usepackage{forest}

% Hyperliens
\usepackage{hyperref}

% TikZ
\usepackage{tikz}
\usetikzlibrary{patterns.meta}
\usetikzlibrary{
  arrows.meta,
  shapes.geometric,
  calc,
  intersections,
  patterns,
  positioning,
  shapes.misc,
  fadings,
  through,
  decorations.pathreplacing
}

\begin{document}
\begin{titlepage}
\newcommand{\doctitle}{\blockchain}

\centering
\vspace*{4cm}
{\huge \textbf{\doctitle}}\vfill

\vspace*{12cm}
\begin{tabular}{|c|c|c|}
\hline
\textbf{Rédacteur} & \textbf{Vérificateur} & \textbf{Approbateur} \\
\hline
& & \\
\hline
\end{tabular}

\vspace{0.5cm}
\noindent\rule{10cm}{0.4pt}
\vspace{1cm}
\begin{flushright}
\begin{center}
Version : 1.0 \hspace{1cm} Titre : \doctitle \hspace{1cm} Réf : DOC-001
\end{center}
\end{flushright}

\end{titlepage}

\newpage
\author{Quentin ANÈS}
\title{Analyse statique}

%\geometry{hmargin=2.5cm,vmargin=2cm}

\pagestyle{fancy}

\definecolor{ColorOne}{named}{MidnightBlue}
\definecolor{ColorTwo}{named}{Dandelion}
\definecolor{ColorThree}{named}{Plum}

\fancyhead[L]{}
\fancyhead[C]{Analyse statique}
\fancyhead[R]{}

\renewcommand{\footrulewidth}{1pt}
\fancyfoot[C]{\text{page \thepage /\pageref{LastPage}}} 
\fancyfoot[L]{\includegraphics[width = 0.08\linewidth]{./images/dowsers_logo.png}}
\fancyfoot[R]{\today}

\begin{center}
\includegraphics[width = 0.5\linewidth]{./images/dowsers_logo_signature.png}
\end{center}

\section{Introduction}
Le projet d'\href{https://fr.wikipedia.org/wiki/Analyse_statique_de_programmes}{analyse statique} repose sur la détection de motif permettant la correspondance avec les codes des smart-contracts\footnote{Les contrats intelligents (en anglais : smart contracts) sont des protocoles informatiques qui facilitent, vérifient et exécutent la négociation ou l'exécution d'un contrat sous forme de code informatique.} étudiés.

\section{Analyse statique}
\subsection{Contexte}
L'analyse statique prend essentiellement en entrée le code source à analyser.
Le langage principal sur lequel les motifs sont élaborées est \href{https://soliditylang.org/}{Solidity}\includegraphics[width = 0.03\linewidth]{./images/solidity_logo.png}.
Cette analyse permet de vérifier l'inclusion de risques dans le code sources à partir d'une base de risques modélisés par Dowsers de manière quasi instantanée.
Cette base de risques est alimentée par la spécification des langages, les retours d'expérience.
\\
\\
\begin{center}
\textbf{Dépôt} : \url{https://github.com/Dowsers/tarkastus_backend}\\
\end{center}

La base de données de risques modélisées à ce jour est présentée dans le tableau ci-dessous:
\begin{center}
\begin{tabular}{|c|c|c|c|c|c|}
\hline 
$Verification\ implementees$ & $H$ & $M$ & $L$ & $G$ & $NC$\\ 
\hline 
64 & 5 & 12 & 13 & 19 & 13\\ 
\hline 
\end{tabular}
\end{center}

\subsection{Objectifs}
Les objectifs portent sur plusieurs points:
\begin{enumerate}
\item Amélioration continue de l'outil.
\item Enrichissement continue de la base de vérifications.
\item Ouverture à d'autre techniques de vérification portant sur le code compilé, métadonnées, arbre de la syntaxe abstraite...
\item Intégration d'un décompilateur \href{https://soliditylang.org/}{Solidity}\includegraphics[width = 0.03\linewidth]{./images/solidity_logo.png}.
\item Ouverture aux langages \href{https://www.rust-lang.org/fr/}{Rust}\includegraphics[width = 0.03\linewidth]{./images/rust-logo-blk.png}, \href{https://www.cairo-lang.org/}{Cairo}\includegraphics[width = 0.03\linewidth]{./images/cairo-logo-square.png}, \href{https://webassembly.org/}{WebAssembly}\includegraphics[width = 0.03\linewidth]{./images/web-assembly-icon-128px.png}.
\end{enumerate}

\newpage
\subsection{Organisation fonctionnelle}
\begin{center}
\begin{forest}
    for tree={draw,edge={Turned Square[open]}-}
    [tarkastus
        [Base de risques (4. Properties), {fill=blue!20}]
        [Smart Contract (1. Collect), for tree={edge=-}
        		[Analyse statique (5. Prove \& 7. Classify)
        			[ Notation standard Dowsers (6. Score)
        				[ Production de rapport ]
        			]
        		]
        ]
    ]
\end{forest}
\end{center}

\subsubsection{tarkastus}
Le module tarkastus est le principale lanceur des différentes analyses. Il agrège et organise les différentes étapes du logiciel.

\subsubsection{Base de risques (4. Properties)}
Le module tarkastus est le principale lanceur des différentes analyses. Il agrège et organise les différentes étapes du logiciel.

\subsection{Planification}
\tikzstyle{descript} = [text = black,align=center, minimum height=1.8cm, align=center, outer sep=0pt,font = \footnotesize]
\tikzstyle{activity} =[align=center,outer sep=1pt]

\begin{tikzpicture}[very thick, black]
\small

%% Coordinates
\coordinate (O) at (-1,0);
\coordinate (P1) at (4,0);
\coordinate (P2) at (8,0);
\coordinate (P3) at (12,0);
\coordinate (F) at (13,0);
\coordinate (E1) at (5,0);
\coordinate (E2) at (0.5,0);

%% Filled regions
\fill[color=ColorOne!20] rectangle (O) -- (P1) -- ($(P1)+(0,1)$) -- ($(O)+(0,1)$);
%\path [pattern color=ColorOne, pattern=north east lines, line width = 1pt, very thick] rectangle ($(O)+(0.5,0)$) -- ($(O)+(2,0)$) -- ($(O)+(2,1)$) -- ($(O)+(0.5,1)$);
\fill[color=ColorTwo!20] rectangle (P1) -- (P2) -- ($(P2)+(0,1)$) -- ($(P1)+(0,1)$);
\shade[left color=ColorThree, right color=white] rectangle (P2) -- (P3) -- ($(P3)+(0,1)$) -- ($(P2)+(0,1)$);

%% Text inside filled regions
\draw ($(P1)+(-2.5,0.5)$) node[activity,ColorOne] {Démonstrateur alpha};
\draw ($(P2)+(-2,0.5)$) node[activity,ColorTwo] {Démonstrateur beta};
\draw ($(P3)+(-2,0.5)$) node[activity,ColorThree] {Perfectionnement};

%% Description
\node[descript,fill=ColorOne!15,text=ColorOne](D1) at ($(P1)+(-2,-2.5)$) {%
    \textbf{Where}\\
    Base de risques (4.Properties)\\
    A.Statique (5.Prove \& 7.Classify)\\
    Notation (6. Score)\\
    Automatisation rapports
};
\node[descript,fill=ColorTwo!15,text=ColorTwo](D2) at ($(P2)+(-2,-2.5)$) {%
    \textbf{Where}\\
    Smart Contract (1. Collect)\\
    Compilation solidity\\
    A.Algorithmique
};
\node[descript,fill=ColorThree!15,text=ColorThree](D3) at ($(P3)+(-2,-2.5)$) {%
    \textbf{Where}\\
    Performance +\\
    Base risques +\\
    Langages +\\
};

%% Events
\draw[<-,thick,color=black] ($(E1)+(0,0.1)$) -- ($(E1)+(0,1.5)$)
  node [above=0pt,align=center,black] {Award Banque de France\\ Campus Cyber};

%% Arrows
\path[->,color=ColorOne] ($(P1)+(0,-0.1)$) edge [out=-90, in=130]  ($(D1)+(0,1)$);
\path[->,color=ColorTwo] ($(P2)+(-1,-0.1)$) edge [out=-90, in=130]  ($(D2)+(0,1)$);
\path[->,color=ColorThree] ($(P3)+(-1,-0.1)$) edge [out=-70, in=90]  ($(D3)+(0,1)$);

%% Arrow
\draw[->] (O) -- (F);

%% Ticks
\foreach \x in {0,2,...,12}
  \draw(\x cm,3pt) -- (\x cm,-3pt);

%% Labels
\foreach \i \j in {0/T2 2023,4/T1 2024,8/T4 2024,12/T2 2025} {
  \draw (\i,0) node[below=3pt] {\j} ;
}
\end{tikzpicture}

\bibliographystyle{alpha}
\bibliography{analyse_statique_bib}

\end{document}
